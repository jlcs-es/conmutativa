\documentclass[10pt,a4paper]{article}
\usepackage[utf8]{inputenc}
\usepackage{amsmath}
\usepackage{amsfonts}
\usepackage{amssymb}
\usepackage{graphicx}
\usepackage{enumerate}
\usepackage[spanish]{babel}
\author{José Luis Cánovas Sánchez}
\title{Álgebra Conmutativa\\Tarea 2: Ejercicios 7-12 Tema 1}
\begin{document}
	\maketitle
	
	\section*{Ejercicio 7}
	
	$\mathcal{P} = \{ x_i  \quad i=1,2,... \quad primos \}  \subseteq \mathbb{Z} $
	
	$\mathbb{Z}_{\left( \mathcal{P} \right)}  = \{ q \in \mathbb{Q} \quad \vert \quad q = \frac{a}{b} \quad a, b \in \mathbb{Z}   \quad (b,x_i) = 1  \, \forall x_i \in \mathcal{P}  \}$
	
	¿$ \mathbb{Z}_{\left( \mathcal{P} \right)} $ subanillo de $\mathbb{Q}$? 
	
	¿$\mathbb{Z}_{\left( \emptyset \right)} $?,  ¿$\mathbb{Z}_{\left( \overline{\mathcal{P}} \right)} $?, donde $\overline{\mathcal{P}}$ es el conjunto de todos los primos.
	
	
	\hfill
	
	Comprobemos la definición 1.5 de subanillo:
	
	$(ii)$ $1\in \mathcal{P}$ 
	
	Como el 1 es coprimo con todos los enteros, $(1, x_i) = 1 \quad \forall x_i \in \mathcal{P}$, tomamos en la representación del racional $q=\frac{a}{b}$ $a=1$ y $b=1$, $q=1\in \mathcal{P}$.

	\hfil
	
	$(i)$ $u, v\in \mathbb{Z}_{\left( \mathcal{P} \right)} \Rightarrow u-v \in \mathbb{Z}_{\left( \mathcal{P} \right)} , \quad  uv \in \mathbb{Z}_{\left( \mathcal{P} \right)} $
	
	Sean $u=\frac{n}{d}$ y $v=\frac{m}{e}$ con $(d,x)=1$ y $(e,x)=1$ $\forall x \in \mathcal{P}$.
	
	$u-v = \frac{n}{d} - \frac{m}{e} = \frac{ne - md}{de}$
	
	$uv = \frac{n}{d}  \frac{m}{e} = \frac{nm}{de}$
	
	Y como $(de, x) = 1 \quad \forall x \in \mathcal{P}$, $u,v\in \mathbb{Z}_{\left( \mathcal{P} \right)} $.
	
	
	\hfill
	
	En el caso $\mathbb{Z}_{\left( \emptyset \right)}$, como no hay elementos en el conjunto $\mathcal{P}$ no hay denominadores posibles que sean coprimos con ellos, ni siquiera el 1, por tanto $\mathbb{Z}_{\left( \emptyset \right)} = \emptyset$.
	
	
	\hfill
	
	Sin embargo, $\mathbb{Z}_{\left( \overline{\mathcal{P}} \right)} $ tiene la propiedad que un denomidador $d$ entero, como se puede factorizar de manera única en producto de primos ($\mathbb{Z}$ es DFU), siempre será múltiplo de alguno, incluso si $d$ es primo, ocurrirá $(d,d)=d$ por $d\in \overline{\mathcal{P}}$. Por tanto el único denominador coprimo con todos los elementos de $\overline{\mathcal{P}}$ es el 1, y por tanto $\mathbb{Z}_{\left( \overline{\mathcal{P}} \right)} = \mathbb{Z}$.
	
	
	\section*{Ejercicio 8}
	
	Describir $\mathbb{Z}_{\left( \mathcal{P} \right)} $ cuando $\mathcal{P}$ está formado por un solo número primo $p$; y encontrar todos sus ideales.
	
	\hfill
	
	
	Los elementos de $\mathbb{Q}$ cuyo denomidador no es múltiplo de $p$. Además tenemos en cuenta que $\mathbb{Z} \subset \mathbb{Z}_{(\mathcal{P})}$:
	
	$ \mathbb{Z} \bigcup \left( \mathbb{Q} \smallsetminus \left( \frac{1}{p} \right) \right) $ 
	
	
	\hfill
	
	Los ideales serán el nulo $(0)$, el trivial $\mathbb{Z}_{\left( \mathcal{P} \right)} $ y un ideal propio sólo podrá estar generado por elementos no unidades (por ejercicio3.ii). El primo $p$ es el único elemento sin inverso, y por tanto el único ideal propio es el $(p)$.
	
	
	\section*{Ejercicio 9}
	
	¿Existe algún anillo que tenga exactamente tres ideales?
	
	
	\hfill
	
	El del ejercicio 8.
	
	\hfill
	
	
	Otro ejemplo lo sacamos pensando en que no puede ser un cuerpo (ejercicio 3.a) y que por simplicidad sea finito, acabamos pensando en los anillos de enteros módulo un número \textbf{no} primo.
	
	Por ejemplo, el anillo de los enteros $\mathbb{Z}_4 = \{0,\, 1,\, 2,\, 3\}$ (siendo $0,\, 1,\, 2,\, 3$ los representantes de las clases módulo 4).
	
	Los ideales posibles son:
	
	$(0) = \{0\} $
	
	$(1) = \mathbb{Z}_4 $
	
	$(2) = \{ 0,\, 2  \} $
	
	\hfil
	
	Cualquier otro ideal da lugar a uno de los anteriores:
	
	$(3) = \{ 0,\, 3,\, 2,\, 1 \} = (1) $
	
	$(\{0,\, 2 \}) = (2)$
	
	$(\{1,\, 2 \}) = (1)$
	
	$(\{2,\, 3 \}) = (1)$
	
	...
	
	
	
	\section*{Ejercicio 10}
	
	
	Demostrar que todo subanillo de $\mathbb{Q} $ es igual a algún $ \mathbb{Z}_{\left( \mathcal{P} \right)} $.
	
	\hfill
	
	Sea $\mathcal{Z}$ el conjunto de todos los $ \mathbb{Z}_{\left( \mathcal{P} \right)} $.
	
	Sea $\mathcal{S}$ el conjunto de todos los subanillos de $\mathbb{Q}$.
	
	\hfill
	
	$\mathcal{S} \supseteq \mathcal{Z}$
	
	Por el ejercicio 7, todo  $ \mathbb{Z}_{\left( \mathcal{P} \right)} $ es subanillo.
	
	
	
	
	\hfill
	
	$\mathcal{S} \subseteq \mathcal{Z}$
	
	Sea $S \in \mathcal{S}$ un subanillo de $\mathbb{Q}$.
	
	Sean $q, t \in S \subset \mathbb{Q}$, con representantes en $\mathbb{Q}$ $q= \frac{a}{b}$ y $t = \frac{c}{d}$, que cumplen $(a,b)=1$ y $(c,d)=1$.
	
	Por $S$ subanillo, $q-t\in S$ y $qt\in S$, y aplicando ambas operaciones hemos visto en Ej.7 que los denominadores son en ambos casos $b \cdot d$.
	
	Si descomponemos todos los denominadores que no son 1 de los elementos de $S$ en producto de primos ($\mathbb{Z}$ DFU), y tomamos su intersección, tenemos un conjunto $\mathcal{P}$ de primos donde todos los denominadores de $S$ son coprimos con todos los elementos de $\mathcal{P}$.
	
	
	
	
	
	\section*{Ejercicio 11}
	
	Demostrar que el conjunto de los ideales no nulos de un dominio tiene una estructura de monoide con el producto de ideales. Probar que si el anillo es un DIP, entonces el monoide es cancelativo.
	
	\hfill
	
	
	Veamos que cumple la asociatividad y tiene elemento neutro (propiedades de ser un monoide):
	
	Sea $\mathcal{I}$ el conjunto de los ideales no nulos $I$ del anillo $A$. No vacío porque el ideal $(1) = A$ siempre existe.
	
	El producto se define como $ IJ = \left( \{ xy \vert x\in I \, y \in J \} \right)  $.
	
	Veamos que el ideal $(1) = A $ es el neutro: ¿ $I A = AI = I $ ?
	
	$ IA = \left( \{ xa \vert x\in I \, a \in A \} \right)  \Rightarrow  $ por $I$ ideal, y $A$ conmutativa por la suposición del inicio del capítulo, $ax = xa \in I$ $ \Rightarrow $ $ IA =  \left( \{ x \vert x\in I \} \right) \Rightarrow IA = AI = I$ .
	
	Sean $I$, $J$ y $K$ tres ideales de $A$.
	
	¿$(IJ)K = I(JK)$?
	
	$(IJ) =  \left( \{ xy \vert x\in I \, y \in J \} \right)  = $ [Prop. 2.11] $ = \{ \sum_j a_j x_j y_j \vert a_j \in A, \; x_j \in I, \; y_j \in J \}$
	
	$(IJ)K =  \left( \{ hz \vert h\in IJ \, z \in K \} \right) =  \{ \sum_j a_j x_j y_j z_j \vert a_j \in A, \; x_j \in I, \; y_j \in J, \; z_j \in K \} $ 
	
	Con la misma construcción para $I(JK)$ tenemos:
	
	$I(JK) =  \left( \{ xu \vert x\in I \, u \in JK \} \right) =  \{ \sum_j a_j x_j y_j z_j \vert a_j \in A, \; x_j \in I, \; y_j \in J, \; z_j \in K \} $ 
	
	$\Rightarrow (IJ)K = I(JK)$.
	
	\hfil
	
	Ahora el caso de que sea $A$ un DIP.
	
	Tomemos $I=(a)$, $J=(b)$, $K=(c)$ ideales principales de $A$, por hipótesis no nulos.
	
	¿$IJ=IK \Rightarrow J=K$?
	
	Sea $a_i \in A$:
	
	$IJ = (a)(b) = \left\langle  \sum_i a_i \cdot ab \right\rangle = \left\langle  \sum_i ai \cdot ac \right\rangle  = IK$
	
	Como los $a_i\cdot a \in A$, podemos considerar los mismos conjuntos generados pero en vez de por $ab$ y $ac$, por $b$ y $c$, lo que nos da que los ideales son los mismos y $J=K$, que es la condición de ser cancelativo.
	
	
	
	
	
	
	\section*{Ejercicio 12}
	
	Aplicar la Proposición 2.22 al anillo de los enteros.
	
	\hfill
	
	
	Como $\mathbb{Z}$ es un DIP, podemos expresar los $I_i$ de la proposición como $(r_i)$ con cada $r_i \in \mathbb{Z}$
	
	La aplicación quedaría:
	
	$ \phi \colon \mathbb{Z} \rightarrow \frac{\mathbb{Z}}{(r_1)} \times \cdots \times \frac{\mathbb{Z}}{(r_n)} $
	
	
	donde $ \phi \left( x  \right) = ( x+(r_1) ,\, x+(r_2), \ldots,\, x+(r_n) )  $
	
	
	Llamando $ x_i $ a los valores tal que $ x \equiv x_i\, mod\, r_i$, nos queda:
	
	$ \phi \left( x  \right) = ( x_1, \ldots, x_n )  $
	
	\hfil 
	
	Si los ideales son primos dos a dos, veamos que siendo principales implica que los $r_i$ también son primos dos a dos:
	
	Si $(r_i)$ y $(r_j)$ son primos, $(1) = (r_i) + (r_j)$, por lo que existen $u, v \in \mathbb{Z} $ tal que $ ur_i + vr_j = 1$.
	
	Esta es la Identidad de Bezout e implica que $r_i$ y $r_j$ son coprimos: $(r_i, r_j) =1 $.
	
	
	\hfil
	
	Además, el producto $\prod (r_i) = (r_1 \cdots r_n) = (N)$ con $N = r_1\cdots r_n$
	
	Tenemos por la proposición 2.22 que existe un isomorfismo 
	
	$\frac{\mathbb{Z}}{N} \cong  \frac{\mathbb{Z}}{(r_1)} \times \cdots \times \frac{\mathbb{Z}}{(r_n)}$
	
	\hfill 

 	En resumen, tenemos el Teorema Chino de los restos:
 	
 	\hfil 
 	
 	Supongamos que $n_1,\, n_2, …, n_k$
 	son enteros positivos coprimos dos a dos. Entonces, para enteros dados $a_1,\, a_2, …, a_k$, existe un
 	entero $x$ que resuelve el sistema de congruencias simultáneas
 	
 	\begin{align*}
 	x &\equiv a_1 \pmod{n_1} \\
 	x &\equiv a_2 \pmod{n_2} \\
 	&\vdots \\
 	x &\equiv a_k \pmod{n_k}
 	\end{align*}
 	
 	Más aún, todas las soluciones $x$ de este sistema son congruentes módulo el
 	producto $N = n_1 n_2 ... n_k$.
 	 	
 	En lenguaje algebraico es que para cada entero positivo con factorización en números primos:
 	
 	$n = p_1^{r_1}\cdots p_k^{r_k}$
 	
 	se tiene un isomorfismo entre un anillo y la suma directa de sus potencias primas:
 	
	$\mathbf{Z}/n\mathbf{Z} \cong \mathbf{Z}/p_1^{r_1}\mathbf{Z} \oplus \cdots \oplus \mathbf{Z}/p_k^{r_k}\mathbf{Z}$
 	
	
\end{document}