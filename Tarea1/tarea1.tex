\documentclass[10pt,a4paper]{article}
\usepackage[utf8]{inputenc}
\usepackage{amsmath}
\usepackage{amsfonts}
\usepackage{amssymb}
\usepackage{graphicx}
\usepackage{enumerate}
\usepackage[spanish]{babel}
\author{José Luis Cánovas Sánchez}
\title{Álgebra Conmutativa\\Tarea 1: Ejercicios 1-6 Tema 1}
\begin{document}
	\maketitle
	
	\section{Ejercicio 1}
	
	Demostrar la proposición 1.11:
	
	\par
	
	\textbf{Proposición 1.11:} \textit{Sea D un dominio. Se tiene:}
	
	\begin{enumerate}[(i)]
		\item \textit{La relación x$\vert$y es transitiva y reflexiva.}
		
		Sean a,b,c,d $\in$ D tal que a$\vert$b con $ad=b$ y b$\vert$c con $be=c$. Sustituyendo b $ade=c$, llamando $f=d\cdot e$ queda $af=c$ por lo que a$\vert$c. La relación es \textbf{transitiva}.
		
		La relación es \textbf{reflexiva} pues según la notación de la definición 1.10 tomando $c=1$, a$\vert$a ya que $a\cdot1=a$.
		
		\item \textit{Si x$\vert$a y x$\vert$b, entonces x|a$\pm$b}
		
		Por $x\vert a$ $\exists c \in D$ tal que $xc=a$. Del mismo modo $\exists d \in D$ tal que $xd=b$.
		
		Si sumamos $xc=a$ y  $xd=b$ nos queda:
		
		  $xd + xd= a + b$ $\Rightarrow$ $x(c+d)=a+b$ $\Rightarrow$ $x\vert a+b$
		  
	    Si restamos $xc=a$ y  $xd=b$ nos queda:
		  
		  $xd - xd= a - b$ $\Rightarrow$ $x(c-d)=a-b$ $\Rightarrow$ $x\vert a-b$
		
	\end{enumerate}
	
	
	\section{Ejercicio 2}
	
	Sea A un anillo, $B\subseteq A$ un subconjunto. Supóngase que las restricciones de las operaciones de A dan operaciones en B y que, con esas operaciones, B es un anillo. Mostrar que B es entonces un subanillo de A si y solo si $1 \in B$.
	
	$\Rightarrow$
	
	Supongamos B subanillo.
	
	Por $(ii)$ de la definición 1.5,se cumple que  $1 \in B$.
	
	\hfill
	
	$\Leftarrow$
	
	Supongamos  $1 \in B$.
	
	De manera inmediata tenemos $(ii)$ de la definición 1.5.
	
	Veamos si $a,\, b \in B \Rightarrow a - b \in B , \, ab \in B$
	
	Por B anillo, $1\in B$ y existe su simétrico $-1$. Además es cerrado para sumas y productos por lo que cumple $ab\in B$ y 
	
	 $b\cdot (-1) = -b \in B$  $\Rightarrow$
	 
	 $a + (-b) = a - b \in B$
	
	\hfill
	
	Como cumple $(i)$ y $(ii)$ de la definición 1.5, B es subanillo.
	
	
	
	
	
	\section{Ejercicio 3}
	
	Sea A un anillo no trivial. Probar:
	
	\begin{enumerate}[(i)]
		\item Si $a\in A$, entonces $(a)=A$ si y solo si $a$ es unidad.
		
		
		
		\item Un ideal de A es propio si y solo si ninguno de sus elementos es una unidad.
		
		
		
		\item A es un cuerpo si y solo si tiene exactamente dos ideales.
		
		
		
		
	\end{enumerate}
	
	
	
	
	
	\section{Ejercicio 4}
	
	Demostrar la Proposición 2.10.
	
	\textbf{Proposición 2.10:} \textit{Sea $A$ un anillo no trivial y sea $C$ un conjunto no vacío de ideales de $A$. El conjunto intersección:}
	\begin{gather*}
		\cap C =\{ x\in A \, \vert \, x\in I \quad \forall I \in C \} 
	\end{gather*}
	 \textit{es un ideal de A.}
	
	
	
	
	
	\section{Ejercicio 5}
	
	Demostrar la Proposición 2.15.
	
	
	\textbf{Proposición 2.15:} \textit{Sea A un anillo no trivial, I un ideal de A, S un subconjunto no vacío de A. El conjunto}
	\begin{gather*}
	\left( I \colon S \right) = \{ x\in A \, \vert \, xS \subseteq I \} 
	\end{gather*}
	\textit{ (donde $ xS = \{ xs \, \vert \, s \in S \} $)	es un ideal de A que contiene a I.}
	
	
	
	
	
	
	
	
	\section{Ejercicio 6}
	
	Sean $A,\, B$ anillos, $f:A\rightarrow B$ una aplicación que cumple las condiciones $(i)$ y $(ii)$ de la Definición 2.1. Probar que $f$ es homomorfismo de anillos si y solo si $f(1)$ es una unidad del anillo $B$.
	
	
	
	
	
	
	
\end{document}