\documentclass[10pt,a4paper]{article}
\usepackage[utf8]{inputenc}
\usepackage{amsmath}
\usepackage{amsfonts}
\usepackage{amssymb}
\usepackage{graphicx}
\usepackage{enumerate}
\usepackage[spanish]{babel}
\author{José Luis Cánovas Sánchez}
\title{Álgebra Conmutativa\\Tarea 1: Ejercicios 1-6 Tema 1}
\begin{document}
	\maketitle
	
	\section{Ejercicio 1}
	
	Demostrar la proposición 1.11:
	
	\par
	
	\textbf{Proposición 1.11:} \textit{Sea D un dominio. Se tiene:}
	
	\begin{enumerate}[(i)]
		\item \textit{La relación x$\vert$y es transitiva y reflexiva.}
		
		Sean a,b,c,d $\in$ D tal que a$\vert$b con $ad=b$ y b$\vert$c con $be=c$. Sustituyendo b $ade=c$, llamando $f=d\cdot e$ queda $af=c$ por lo que a$\vert$c. La relación es \textbf{transitiva}.
		
		La relación es \textbf{reflexiva} pues según la notación de la definición 1.10 tomando $c=1$, a$\vert$a ya que $a\cdot1=a$.
		
		\item \textit{Si x$\vert$a y x$\vert$b, entonces x|a$\pm$b}
		
		Por $x\vert a$ $\exists c \in D$ tal que $xc=a$. Del mismo modo $\exists d \in D$ tal que $xd=b$.
		
		Si sumamos $xc=a$ y  $xd=b$ nos queda:
		
		  $xd + xd= a + b$ $\Rightarrow$ $x(c+d)=a+b$ $\Rightarrow$ $x\vert a+b$
		  
	    Si restamos $xc=a$ y  $xd=b$ nos queda:
		  
		  $xd - xd= a - b$ $\Rightarrow$ $x(c-d)=a-b$ $\Rightarrow$ $x\vert a-b$
		
	\end{enumerate}
	
	
	\section{Ejercicio 2}
	
	Sea A un anillo, $B\subseteq A$ un subconjunto. Supóngase que las restricciones de las operaciones de A dan operaciones en B y que, con esas operaciones, B es un anillo. Mostrar que B es entonces un subanillo de A si y solo si $1 \in B$.
	
	$\Rightarrow$
	
	Supongamos B subanillo.
	
	Por $(ii)$ de la definición 1.5,se cumple que  $1 \in B$.
	
	\hfill
	
	$\Leftarrow$
	
	Supongamos  $1 \in B$.
	
	De manera inmediata tenemos $(ii)$ de la definición 1.5.
	
	Veamos si $a,\, b \in B \Rightarrow a - b \in B , \, ab \in B$
	
	Por B anillo, $1\in B$ y existe su simétrico $-1$. Además es cerrado para sumas y productos por lo que cumple $ab\in B$ y 
	
	 $b\cdot (-1) = -b \in B$  $\Rightarrow$
	 
	 $a + (-b) = a - b \in B$
	
	\hfill
	
	Como cumple $(i)$ y $(ii)$ de la definición 1.5, B es subanillo.
	
	
	
	
	
	\section{Ejercicio 3}
	
	Sea A un anillo no trivial. Probar:
	
	\begin{enumerate}[(i)]
		\item Si $a\in A$, entonces $(a)=A$ si y solo si $a$ es unidad.
		
		
		$\Rightarrow$
		
		Si $(a)=A$ el caso particular $1\in A = (a)$ y por la definición de ideal principal $(a) = \{ ax \, \vert \, x\in A \}$ entonces $\exists b \in A$ tal que $b\cdot a = 1 \in (a) = A$ por lo que $a$ es unidad.
		
		\hfill
		
		$\Leftarrow$
		
		Sea $a$ unidad. $\exists b \in A$ tal que $ab=1$.
		
		$(a) = \{ xa \, \vert \, x\in A \}$ y tomando $x=b$ $ba=1 \in (a)$
	
		$1 \in (a) \Rightarrow \forall x \in A \quad x\cdot 1 = x \in (a) \Rightarrow (a) = A$	
		
		\hfill
		
		
		\item Un ideal de A es propio si y solo si ninguno de sus elementos es una unidad.
		
		$\Rightarrow$
		
		Sea $I \trianglelefteq A$ ideal cualquiera de A.
		Supongamos  que $\exists a \in I$ unidad, i.e. $\exists b \in A$ con $ab=1$ $\Rightarrow 1\in I \Rightarrow  \forall x \in A \quad x\cdot 1 = x \in I \Rightarrow A = I $
		
		Por la equivalencia de $A \Rightarrow B$ $\equiv$ $\neg B \Rightarrow \neg A$, tenemos demostrada la implicación a la derecha.
		
		
		
		
		\hfill
		
		
		$\Leftarrow$
		
		Supongamos que $I \trianglelefteq A$ ideal impropio $I=A$. Como $1 \in A$ unidad ($1\cdot 1 = 1$), tenemos demostrada la implicación izquierda. 
		
		
		
		\hfill
		
		
		\item A es un cuerpo si y solo si tiene exactamente dos ideales.
		
		$\Rightarrow$
		
		Por ser $A$ cuerpo, todo elemento no nulo $a\in A$ tiene inverso $b\in A$ tal que $ab=1$.
		
		Sea $I \trianglelefteq A$ un ideal cualquiera de $A$. Para toda unidad $a$,  $a \in I$, $ab \in I \Rightarrow 1 \in I \Rightarrow I=A$. Todo ideal con algún elemento no nulo es ideal impropio.
		
		Si $I$ no tiene unidades, como no puede ser vacío, sólo queda $0 \in I$, por lo que $I = (0)$ el ideal trivial.
		
		$A$ tiene exactamente dos ideales.
		
				
		\hfill
		
		
		$\Leftarrow$
		
		Si todo ideal no trivial es el impropio $I=A$. Tomamos un elemento arbitrario $a \in A$, de modo que su ideal principal será $(a) = A$.
		
		Suponemos por reducción al absurdo que $A$ no es un cuerpo.
		
		$\exists x \in A$ no nulo $\colon \forall y \in A \quad x\cdot y \neq 1$, i.e., no es unidad.
		
		Como todo ideal principal $(a) = A$, entonces $(x) = A$, pero $1 \in A$ y $1 \notin (x) = A$. Contradicción con la hipótesis de que $A$ no es cuerpo.
		
		
		\hfill
		
		
	\end{enumerate}
	
	
	
	
	
	\section{Ejercicio 4}
	
	Demostrar la Proposición 2.10.
	
	\textbf{Proposición 2.10:} \textit{Sea $A$ un anillo no trivial y sea $C$ un conjunto no vacío de ideales de $A$. El conjunto intersección:}
	\begin{gather*}
		\cap C =\{ x\in A \, \vert \, x\in I \quad \forall I \in C \} 
	\end{gather*}
	 \textit{es un ideal de A.}
	
	
	\hfill
	
	Veamos los puntos $(i)$ y $(ii)$ de la definición 2.2 de ideal:
	
	$(i)$ $x,y\in \cap C \Rightarrow x+y \in \cap C$
	
	Si $x, y \in \cap C$, entonces $x, y \in I \quad \forall I \in C \Rightarrow x+y \in I$ por $I$ ideal $\forall I \in C \Rightarrow x+y \in \cap C $.
	
	\hfill
	
	$(ii)$ $x \in \cap C \quad a \in A \Rightarrow ax \in \cap C$
	
	Si $x \in \cap C$, entonces $x \in I \quad \forall I \in C \Rightarrow xa\in I$ por $I$ ideal $\forall I \in C \Rightarrow xa \in \cap C $
	
	\section{Ejercicio 5}
	
	Demostrar la Proposición 2.15.
	
	
	\textbf{Proposición 2.15:} \textit{Sea A un anillo no trivial, I un ideal de A, S un subconjunto no vacío de A. El conjunto}
	\begin{gather*}
	\left( I \colon S \right) = \{ x\in A \, \vert \, xS \subseteq I \} 
	\end{gather*}
	\textit{ (donde $ xS = \{ xs \, \vert \, s \in S \} $)	es un ideal de A que contiene a I.}
	
	
	\hfill
	
	Veamos que es ideal:
	
	\hfill
	
	$(i)$ $a,b\in \left( I \colon S \right)  \Rightarrow a+b \in \left( I \colon S \right)$
	
	Por $a \in \left( I \colon S \right)$ se cumple $\forall s \in S \quad a\cdot s \in I$.
	
	Análogamente $\forall s \in S \quad b\cdot s \in I$.
	
	Sumamos: $as + bs \in I$ por ser elementos del ideal $I$.
	
	Por la propiedad distributiva $as+bs=(a+b)s \in I$ cumple la definición para el elemento $a+b \in \left( I \colon S \right).$
	
	\hfill
	
	$(ii)$ $a \in \left( I \colon S \right) \quad x \in A \Rightarrow ax \in \left( I \colon S \right)$
	
	Por $a \in \left( I \colon S \right)$ teníamos $\forall s \in S \quad a\cdot s \in I$.
	
	Por $I$ ideal $x\cdot as \in I \left( \forall s \in S  \right) \Rightarrow xa \in \left( I \colon S \right)$ por definición de $\left( I \colon S \right)$.
	
	
	
	\hfill
	
	Veamos ahora que $I \subseteq \left( I \colon S \right) $:
	
	Sea $x\in I $ arbitrario.
	
	$x s \in I \quad \forall s \in S $ por $ S \subseteq A $ e $I$ ideal.
	
	$x \in \left( I \colon S \right) \quad \forall x \in I$ por definición de $\left( I \colon S \right)$
	
	$I \subseteq \left( I \colon S \right) $
	
	
	\section{Ejercicio 6}
	
	Sean $A,\, B$ anillos, $f:A\rightarrow B$ una aplicación que cumple las condiciones $(i)$ y $(ii)$ de la Definición 2.1. Probar que $f$ es homomorfismo de anillos si y solo si $f(1)$ es una unidad del anillo $B$.
	
	\hfill
	
	$\Rightarrow$
	
	Inmediata.
	
	Por $f$ homomosfismo cumple 2.1.$(iii)$: $f(1_A) = 1_B \in U\left( B \right)$
	
	
	\hfill
	
	
	$\Leftarrow$
	
	Sean $u,v\in U(B)$ tal que $uv=1_B$ y $f(1_A)=u$.
	
	 Si $f$ fuera homomorfismo de anillos debería cumplir $(ii)$ y $(iii)$, lo que  induce un homomorfismo de grupos multiplicativos $f \colon U(A) \rightarrow U(B)$, que en particular cumple $ f(u^{-1})=f(u)^{-1} \quad \forall u \in U(A)$
	 
	 $f(1) = f(1^{-1}) = f(1)^{-1} = u ^{-1} = v $
	 
	 $u = f(1) = v \Rightarrow U(B) = \{1_B\} \Rightarrow f(1_A) = 1_B$
	 
	 Hemos encontrado una restricción extra necesaria en $B$ para que $f$ pueda ser homomorfismo de anillos, todas sus unidades son iguales, por lo que serán el $1$. 
	
	
	\hfill
	
	
	
\end{document}